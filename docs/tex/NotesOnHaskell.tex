\documentclass[11pt, a4paper]{article}

 \usepackage[left=2.5cm, right=2.5cm, top=2cm, bottom=2.5cm]{geometry}
\usepackage{amsmath}
\usepackage{amssymb}
\usepackage{amsfonts}

\usepackage{hyperref}
\usepackage{tikz}

\newcommand{\ExternalLink}{%
    \tikz[x=1.2ex, y=1.2ex, baseline=-0.05ex]{% 
        \begin{scope}[x=1ex, y=1ex]
            \clip (-0.1,-0.1) 
                --++ (-0, 1.2) 
                --++ (0.6, 0) 
                --++ (0, -0.6) 
                --++ (0.6, 0) 
                --++ (0, -1);
            \path[draw, 
                line width = 0.5, 
                rounded corners=0.5] 
                (0,0) rectangle (1,1);
        \end{scope}
        \path[draw, line width = 0.5] (0.5, 0.5) 
            -- (1, 1);
        \path[draw, line width = 0.5] (0.6, 1) 
            -- (1, 1) -- (1, 0.6);
        }
    }
 
 \newcommand{\Chi}{\mathrm{X}}

% Define the new "Problem" environment
\newtheorem{problem}{Problem}[subsection]

\begin{document}

\title{Notes on Haskell}

\author{Tyler Foster}

\maketitle

\begin{section}{From category theory to Haskell}

\begin{subsection}{Types versus sets-with-structure}
[...]
\end{subsection}

\begin{subsection}{Comparing $\mathbf{Hask}_\text{tot}$ to $\bold{Sets}$}
\end{subsection}

\begin{subsection}{The category $\text{IO}(\mathbf{Hask}_\text{tot})$.}
[...]
\end{subsection}

\begin{subsection}{The Kleisli category $\mathbf{Kl}_\text{IO}$.}
[...]
\end{subsection}

\end{section}

\cite{NeoRie}
\cite{Geo}
\cite{Cohn}


\bibliographystyle{alpha}
\bibliography{./bib/dissig}

\end{document}


