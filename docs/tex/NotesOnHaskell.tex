\documentclass[11pt, a4paper]{article}

 \usepackage[left=2.5cm, right=2.5cm, top=2cm, bottom=2.5cm]{geometry}
\usepackage{amsmath}
\usepackage{amssymb}
\usepackage{amsfonts}

\usepackage{hyperref}
\usepackage{tikz}

\newcommand{\ExternalLink}{%
    \tikz[x=1.2ex, y=1.2ex, baseline=-0.05ex]{% 
        \begin{scope}[x=1ex, y=1ex]
            \clip (-0.1,-0.1) 
                --++ (-0, 1.2) 
                --++ (0.6, 0) 
                --++ (0, -0.6) 
                --++ (0.6, 0) 
                --++ (0, -1);
            \path[draw, 
                line width = 0.5, 
                rounded corners=0.5] 
                (0,0) rectangle (1,1);
        \end{scope}
        \path[draw, line width = 0.5] (0.5, 0.5) 
            -- (1, 1);
        \path[draw, line width = 0.5] (0.6, 1) 
            -- (1, 1) -- (1, 0.6);
        }
    }
 
 \newcommand{\Chi}{\mathrm{X}}

% Define the new "Problem" environment
\newtheorem{problem}{Problem}[subsection]

\begin{document}

\title{Notes on Haskell}

\author{Tyler Foster}

\maketitle

\tableofcontents

\begin{section}{From category theory to Haskell}

\begin{subsection}{Types versus sets-with-added-structure}
One major source of confusion when trying to understand Haskell from a category theoretical perspective is the role that {\em types} play in Haskell.

In many ways, category theory overrides the Bourbakian picture of mathematics, in which everything is build up from {\em sets with added structure} \cite{ToolObject} \cite{Geometric}. Still, sets with added structure do form the most familiar examples of categories, and a large part of Haskell focusses on categories that arise from sets with added structure.

But Haskell discusses this notion of ``sets with added structure'' using the language of types, not the language of set theory. My rough heuristic for how to understand the two perspectives is:
\begin{center}
The {\em type} associated to a {\em set} is like the predicate that defines the set.
\end{center}
\begin{center}
The {\em set} associated to a {\em type} is the set of all instances of that type.
\end{center}

\noindent
Think about how things work in an object-oriented scripting language like JavaScript or Python: the user defines new types by describing what internal structure and internal constraints any new instance of this type must have. It's a recipe for how new elements of some not-yet-fully-populated set will come into being.

To give some concrete examples of what this distinction might mean from a mathematical perspective, [...]

\begin{center}

\begin{tabular}{c|c}
{\bf type} & {\bf set}
\\[4pt]
\hline\hline
\\[-10pt]
nonnegative integer less than $n$ & $\{0,1,2,\dots,n-1\}$
\\[4pt]
\hline
\\[-10pt]
$n$-entry real coordinate vector
&
$\mathbb{R}^n$
\\[4pt]
\hline
\\[-10pt]
prime ideal in $\mathbb{C}[x_1,x_2,\dots,x_n]$
&
$\mathbb{A}^{\!n}_{\mathbb{C}}$
\\[4pt]
\hline
\\[-10pt]
$m\!\times\!n$-matrix with entries in $\mathbb{F}_{\!q}$
&
$\text{Mat}_{m,n}(\mathbb{F}_{\!q})$
\\[4pt]
\hline
\\[-10pt]
list of characters
&
$\{\text{all characters}\}\ =\ \overset{\infty}{\underset{n=0}{\bigsqcup}}\text{Char}^{\times n}$
\\[4pt]
\hline
\\[-10pt]
program executable on present machine
&
$\{\!$
programs executable on present machine
$\!\}$
\\[4pt]
\end{tabular}

\end{center}

\noindent
This distinction might seem a bit odd from a purely mathematical perspective, but it makes more sense from a computer science perspective. In most computing environments, it is {\em much} easier to specify a condition approximating ``$n$-entry real coordinate vector'' than it is to instantiate the set $\mathbb{R}^{n}$.

[...]

\begin{subsubsection}{What types and sets-with-added-structure reject from one another.}
[...]

\begin{enumerate}
\item
Sets with ``hard-to-know'' defining conditions
\item
Functions without defining formulae.
\end{enumerate}

[...]

\begin{enumerate}
\item
Distinct proofs of identity
\item
Recursive types?
\end{enumerate}

\end{subsubsection}

\end{subsection}

\begin{subsection}{Comparing $\mathbf{Hask}_\text{tot}$ to $\bold{Sets}$}
\end{subsection}

\begin{subsection}{The category $\text{IO}(\mathbf{Hask}_\text{tot})$.}
[...]

\begin{subsubsection}{Morphisms in $\text{IO}(\mathbf{Hask}_{\text{tot}})$ as program-level combinators}
\begin{enumerate}
\item
Mapping the result (\texttt{fmap}, \texttt{>>=})
\item
Replacing the behavior (\texttt{const}, \texttt{ioB})
\item
Wrapping the behavior (\texttt{try}, \texttt{catch}, \texttt{log})
\item
Scheduling or deferring execution
\item
Composing effects (\texttt{StateT}, \texttt{ReaderT})
\end{enumerate}
\end{subsubsection}

\end{subsection}

\begin{subsection}{The Kleisli category $\mathbf{Kl}_\text{IO}$.}
[...]
\end{subsection}

\end{section}


\bibliographystyle{alpha}
\bibliography{./bib/CatQuot}

\end{document}


